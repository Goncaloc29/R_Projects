% Options for packages loaded elsewhere
\PassOptionsToPackage{unicode}{hyperref}
\PassOptionsToPackage{hyphens}{url}
%
\documentclass[
]{article}
\usepackage{amsmath,amssymb}
\usepackage{iftex}
\ifPDFTeX
  \usepackage[T1]{fontenc}
  \usepackage[utf8]{inputenc}
  \usepackage{textcomp} % provide euro and other symbols
\else % if luatex or xetex
  \usepackage{unicode-math} % this also loads fontspec
  \defaultfontfeatures{Scale=MatchLowercase}
  \defaultfontfeatures[\rmfamily]{Ligatures=TeX,Scale=1}
\fi
\usepackage{lmodern}
\ifPDFTeX\else
  % xetex/luatex font selection
\fi
% Use upquote if available, for straight quotes in verbatim environments
\IfFileExists{upquote.sty}{\usepackage{upquote}}{}
\IfFileExists{microtype.sty}{% use microtype if available
  \usepackage[]{microtype}
  \UseMicrotypeSet[protrusion]{basicmath} % disable protrusion for tt fonts
}{}
\makeatletter
\@ifundefined{KOMAClassName}{% if non-KOMA class
  \IfFileExists{parskip.sty}{%
    \usepackage{parskip}
  }{% else
    \setlength{\parindent}{0pt}
    \setlength{\parskip}{6pt plus 2pt minus 1pt}}
}{% if KOMA class
  \KOMAoptions{parskip=half}}
\makeatother
\usepackage{xcolor}
\usepackage[margin=1in]{geometry}
\usepackage{color}
\usepackage{fancyvrb}
\newcommand{\VerbBar}{|}
\newcommand{\VERB}{\Verb[commandchars=\\\{\}]}
\DefineVerbatimEnvironment{Highlighting}{Verbatim}{commandchars=\\\{\}}
% Add ',fontsize=\small' for more characters per line
\usepackage{framed}
\definecolor{shadecolor}{RGB}{248,248,248}
\newenvironment{Shaded}{\begin{snugshade}}{\end{snugshade}}
\newcommand{\AlertTok}[1]{\textcolor[rgb]{0.94,0.16,0.16}{#1}}
\newcommand{\AnnotationTok}[1]{\textcolor[rgb]{0.56,0.35,0.01}{\textbf{\textit{#1}}}}
\newcommand{\AttributeTok}[1]{\textcolor[rgb]{0.13,0.29,0.53}{#1}}
\newcommand{\BaseNTok}[1]{\textcolor[rgb]{0.00,0.00,0.81}{#1}}
\newcommand{\BuiltInTok}[1]{#1}
\newcommand{\CharTok}[1]{\textcolor[rgb]{0.31,0.60,0.02}{#1}}
\newcommand{\CommentTok}[1]{\textcolor[rgb]{0.56,0.35,0.01}{\textit{#1}}}
\newcommand{\CommentVarTok}[1]{\textcolor[rgb]{0.56,0.35,0.01}{\textbf{\textit{#1}}}}
\newcommand{\ConstantTok}[1]{\textcolor[rgb]{0.56,0.35,0.01}{#1}}
\newcommand{\ControlFlowTok}[1]{\textcolor[rgb]{0.13,0.29,0.53}{\textbf{#1}}}
\newcommand{\DataTypeTok}[1]{\textcolor[rgb]{0.13,0.29,0.53}{#1}}
\newcommand{\DecValTok}[1]{\textcolor[rgb]{0.00,0.00,0.81}{#1}}
\newcommand{\DocumentationTok}[1]{\textcolor[rgb]{0.56,0.35,0.01}{\textbf{\textit{#1}}}}
\newcommand{\ErrorTok}[1]{\textcolor[rgb]{0.64,0.00,0.00}{\textbf{#1}}}
\newcommand{\ExtensionTok}[1]{#1}
\newcommand{\FloatTok}[1]{\textcolor[rgb]{0.00,0.00,0.81}{#1}}
\newcommand{\FunctionTok}[1]{\textcolor[rgb]{0.13,0.29,0.53}{\textbf{#1}}}
\newcommand{\ImportTok}[1]{#1}
\newcommand{\InformationTok}[1]{\textcolor[rgb]{0.56,0.35,0.01}{\textbf{\textit{#1}}}}
\newcommand{\KeywordTok}[1]{\textcolor[rgb]{0.13,0.29,0.53}{\textbf{#1}}}
\newcommand{\NormalTok}[1]{#1}
\newcommand{\OperatorTok}[1]{\textcolor[rgb]{0.81,0.36,0.00}{\textbf{#1}}}
\newcommand{\OtherTok}[1]{\textcolor[rgb]{0.56,0.35,0.01}{#1}}
\newcommand{\PreprocessorTok}[1]{\textcolor[rgb]{0.56,0.35,0.01}{\textit{#1}}}
\newcommand{\RegionMarkerTok}[1]{#1}
\newcommand{\SpecialCharTok}[1]{\textcolor[rgb]{0.81,0.36,0.00}{\textbf{#1}}}
\newcommand{\SpecialStringTok}[1]{\textcolor[rgb]{0.31,0.60,0.02}{#1}}
\newcommand{\StringTok}[1]{\textcolor[rgb]{0.31,0.60,0.02}{#1}}
\newcommand{\VariableTok}[1]{\textcolor[rgb]{0.00,0.00,0.00}{#1}}
\newcommand{\VerbatimStringTok}[1]{\textcolor[rgb]{0.31,0.60,0.02}{#1}}
\newcommand{\WarningTok}[1]{\textcolor[rgb]{0.56,0.35,0.01}{\textbf{\textit{#1}}}}
\usepackage{graphicx}
\makeatletter
\def\maxwidth{\ifdim\Gin@nat@width>\linewidth\linewidth\else\Gin@nat@width\fi}
\def\maxheight{\ifdim\Gin@nat@height>\textheight\textheight\else\Gin@nat@height\fi}
\makeatother
% Scale images if necessary, so that they will not overflow the page
% margins by default, and it is still possible to overwrite the defaults
% using explicit options in \includegraphics[width, height, ...]{}
\setkeys{Gin}{width=\maxwidth,height=\maxheight,keepaspectratio}
% Set default figure placement to htbp
\makeatletter
\def\fps@figure{htbp}
\makeatother
\setlength{\emergencystretch}{3em} % prevent overfull lines
\providecommand{\tightlist}{%
  \setlength{\itemsep}{0pt}\setlength{\parskip}{0pt}}
\setcounter{secnumdepth}{-\maxdimen} % remove section numbering
\ifLuaTeX
  \usepackage{selnolig}  % disable illegal ligatures
\fi
\IfFileExists{bookmark.sty}{\usepackage{bookmark}}{\usepackage{hyperref}}
\IfFileExists{xurl.sty}{\usepackage{xurl}}{} % add URL line breaks if available
\urlstyle{same}
\hypersetup{
  pdftitle={R Notebook},
  hidelinks,
  pdfcreator={LaTeX via pandoc}}

\title{R Notebook}
\author{}
\date{\vspace{-2.5em}}

\begin{document}
\maketitle

This is an \href{http://rmarkdown.rstudio.com}{R Markdown} Notebook.
When you execute code within the notebook, the results appear beneath
the code.

Try executing this chunk by clicking the \emph{Run} button within the
chunk or by placing your cursor inside it and pressing
\emph{Ctrl+Shift+Enter}.

\begin{Shaded}
\begin{Highlighting}[]
\CommentTok{\# install.packages("dplyr")}
\FunctionTok{library}\NormalTok{(dplyr)}
\end{Highlighting}
\end{Shaded}

\begin{verbatim}
## Warning: package 'dplyr' was built under R version 4.3.1
\end{verbatim}

\begin{verbatim}
## 
## Attaching package: 'dplyr'
\end{verbatim}

\begin{verbatim}
## The following objects are masked from 'package:stats':
## 
##     filter, lag
\end{verbatim}

\begin{verbatim}
## The following objects are masked from 'package:base':
## 
##     intersect, setdiff, setequal, union
\end{verbatim}

\begin{Shaded}
\begin{Highlighting}[]
\NormalTok{data }\OtherTok{\textless{}{-}} \FunctionTok{read.csv}\NormalTok{(}\StringTok{"Spotify\_Song\_Attributes.csv"}\NormalTok{)}
\NormalTok{desiredColumns }\OtherTok{\textless{}{-}} \FunctionTok{c}\NormalTok{(}\StringTok{"trackName"}\NormalTok{, }\StringTok{"artistName"}\NormalTok{, }\StringTok{"msPlayed"}\NormalTok{, }\StringTok{"genre"}\NormalTok{, }\StringTok{"duration\_ms"}\NormalTok{)}
\NormalTok{data }\OtherTok{\textless{}{-}}\NormalTok{ data[, desiredColumns]}
\end{Highlighting}
\end{Shaded}

\begin{Shaded}
\begin{Highlighting}[]
\NormalTok{data }\OtherTok{\textless{}{-}}\NormalTok{ data }\SpecialCharTok{\%\textgreater{}\%}
  \FunctionTok{mutate\_all}\NormalTok{(}\AttributeTok{.funs =} \FunctionTok{list}\NormalTok{(}\SpecialCharTok{\textasciitilde{}}\FunctionTok{ifelse}\NormalTok{(. }\SpecialCharTok{\%in\%} \FunctionTok{c}\NormalTok{(}\StringTok{""}\NormalTok{, }\StringTok{" "}\NormalTok{), }\ConstantTok{NA}\NormalTok{, .)))}
\end{Highlighting}
\end{Shaded}

\begin{Shaded}
\begin{Highlighting}[]
\CommentTok{\# Count the number of rows with at least one NA value}
\NormalTok{lines\_with\_na }\OtherTok{\textless{}{-}} \FunctionTok{sum}\NormalTok{(}\FunctionTok{apply}\NormalTok{(}\FunctionTok{is.na}\NormalTok{(data), }\DecValTok{1}\NormalTok{, any))}
\end{Highlighting}
\end{Shaded}

\begin{Shaded}
\begin{Highlighting}[]
\NormalTok{data }\OtherTok{\textless{}{-}} \FunctionTok{na.omit}\NormalTok{(data)}
\end{Highlighting}
\end{Shaded}

\begin{Shaded}
\begin{Highlighting}[]
\CommentTok{\# Escolha a coluna em que deseja contar as ocorrências dos nomes}
\NormalTok{coluna\_escolhida }\OtherTok{\textless{}{-}} \StringTok{"trackName"}  \CommentTok{\# Substitua "nomeda\_coluna" pelo nome da coluna de interesse}

\CommentTok{\# Remover registros duplicados e manter apenas o primeiro de cada valor repetido}
\NormalTok{data }\OtherTok{\textless{}{-}} \FunctionTok{subset}\NormalTok{(data, }\SpecialCharTok{!}\FunctionTok{duplicated}\NormalTok{(data[[coluna\_escolhida]]))}
\end{Highlighting}
\end{Shaded}

\begin{Shaded}
\begin{Highlighting}[]
\CommentTok{\# Agrupar os dados por artista e calcular a soma dos minutos ouvidos para cada artista}
\NormalTok{artistas\_mais\_ouvidos }\OtherTok{\textless{}{-}}\NormalTok{ data }\SpecialCharTok{\%\textgreater{}\%}
  \FunctionTok{group\_by}\NormalTok{(artistName) }\SpecialCharTok{\%\textgreater{}\%}
  \FunctionTok{summarise}\NormalTok{(}\AttributeTok{msPlayed =} \FunctionTok{sum}\NormalTok{(msPlayed)) }\SpecialCharTok{\%\textgreater{}\%}
  \FunctionTok{ungroup}\NormalTok{()  }\CommentTok{\# Desfazer o agrupamento}

\CommentTok{\# Encontrar o índice do artista com a maior soma de minutos ouvidos}
\NormalTok{indice\_mais\_ouvido }\OtherTok{\textless{}{-}} \FunctionTok{which.max}\NormalTok{(artistas\_mais\_ouvidos}\SpecialCharTok{$}\NormalTok{msPlayed)}

\CommentTok{\# Obter o nome do artista mais ouvido}
\NormalTok{artista\_mais\_ouvido }\OtherTok{\textless{}{-}}\NormalTok{ artistas\_mais\_ouvidos}\SpecialCharTok{$}\NormalTok{artistName[indice\_mais\_ouvido]}

\CommentTok{\# Exibir o resultado}
\FunctionTok{print}\NormalTok{(artista\_mais\_ouvido)}
\end{Highlighting}
\end{Shaded}

\begin{verbatim}
## [1] "RADWIMPS"
\end{verbatim}

\begin{Shaded}
\begin{Highlighting}[]
\CommentTok{\# Ordenar em ordem decrescente com base na soma dos minutos ouvidos}
\NormalTok{artistas\_mais\_ouvidos }\OtherTok{\textless{}{-}}\NormalTok{ artistas\_mais\_ouvidos }\SpecialCharTok{\%\textgreater{}\%}
  \FunctionTok{arrange}\NormalTok{(}\FunctionTok{desc}\NormalTok{(msPlayed))}

\CommentTok{\# Adicionar o número do ranking}
\NormalTok{artistas\_mais\_ouvidos }\OtherTok{\textless{}{-}}\NormalTok{ artistas\_mais\_ouvidos }\SpecialCharTok{\%\textgreater{}\%}
  \FunctionTok{mutate}\NormalTok{(}\AttributeTok{ranking =} \FunctionTok{row\_number}\NormalTok{())}

\CommentTok{\# Selecionar os 10 artistas mais ouvidos e reorganizar as colunas}
\NormalTok{top\_10\_artistas\_mais\_ouvidos }\OtherTok{\textless{}{-}} \FunctionTok{head}\NormalTok{(artistas\_mais\_ouvidos, }\DecValTok{10}\NormalTok{) }\SpecialCharTok{\%\textgreater{}\%}
  \FunctionTok{select}\NormalTok{(ranking, artistName, msPlayed)}

\CommentTok{\# Exibir o resultado}
\FunctionTok{print}\NormalTok{(top\_10\_artistas\_mais\_ouvidos)}
\end{Highlighting}
\end{Shaded}

\begin{verbatim}
## # A tibble: 10 x 3
##    ranking artistName        msPlayed
##      <int> <chr>                <int>
##  1       1 RADWIMPS         290650836
##  2       2 Mokita           188704388
##  3       3 Blake Rose       114545719
##  4       4 Yoh kamiyama     113093301
##  5       5 Lauv             101256022
##  6       6 Sasha Alex Sloan  97431379
##  7       7 Hans Zimmer       95592170
##  8       8 Kygo              84961260
##  9       9 why mona          84185054
## 10      10 Lund              83079142
\end{verbatim}

Add a new chunk by clicking the \emph{Insert Chunk} button on the
toolbar or by pressing \emph{Ctrl+Alt+I}.

When you save the notebook, an HTML file containing the code and output
will be saved alongside it (click the \emph{Preview} button or press
\emph{Ctrl+Shift+K} to preview the HTML file).

The preview shows you a rendered HTML copy of the contents of the
editor. Consequently, unlike \emph{Knit}, \emph{Preview} does not run
any R code chunks. Instead, the output of the chunk when it was last run
in the editor is displayed.

\end{document}
